\documentclass[compress,aspectratio=169]{beamer}
\usepackage{ragged2e}
\usepackage{underscore}
%Setup File
% \usepackage[english]{babel}
\usepackage[brazil]{babel}
\usepackage[utf8]{inputenc}
\usepackage[backend=biber,style=numeric-comp,sorting=none]{biblatex}
\usepackage{comment}
\usepackage{csquotes}
\usepackage{kpfonts}
\usepackage{amsmath}
\usepackage{tcolorbox}


\usepackage{listings}
\lstset{
  language=Python,
  frame=single,  % Adds a frame around the code
  basicstyle=\ttfamily,  % Use typewriter font for code
  keywordstyle=\color{blue},  % Keywords in blue
  commentstyle=\color{gray},  % Comments in gray
  stringstyle=\color{red},  % Strings in red
  breaklines=true  % Automatically break long lines
}


%imprimir 4 por página

\usepackage{pgfpages}
\selectlanguage{portuguese}
%\pgfpagesuselayout{4 on 1}[a4paper,border shrink=5mm,landscape]
% 4 quadros

%\addbibresource{bibliografia.bib}

% 1x1 === Para os alunos === Sem pausas  % http://tex.stackexchange.com/questions/1423/is-there-a-nice-way-to-compile-a-beamer-presentation-without-the-pauses
%  \documentclass[handout]{beamer}

% 3x1 === Para os alunos === Com anotações
%  \documentclass[handout]{beamer}
%  \usepackage{handoutWithNotes}
%  \pgfpagesuselayout{3 on 1 with notes}[a4paper,border shrink=5mm]

% 8x1
%  \documentclass[handout]{beamer}
%  \usepackage{pgfpages}
%  \mode<handout>{\pgfpagesuselayout{8 on 1}[a4paper,border shrink=5mm]}%,landscape]}

% Widescreen
%  \documentclass[aspectratio = 169]{beamer}


%\usepackage{beamerthemeAmsterdam}

% Setup appearance:
\usetheme{Dresden}
\usepackage[onelanguage,ruled,lined,linesnumbered]{algorithm2e}
\SetAlFnt{\small}
\SetNoFillComment
\newcommand\mycommfont[1]{\textit{\footnotesize\ttfamily\textcolor{gray}{#1}}}
\SetCommentSty{mycommfont}
\usepackage{algorithmic}
\algsetup{linenosize=\tiny}
\SetKw{KwBy}{by}
\SetKwRepeat{Do}{do}{while}%

%portuguese

%\defbeamertemplate*{headline}{nosections}{\vskip\headheight}
\usefonttheme[onlylarge]{structurebold}
\setbeamerfont*{frametitle}{size=\normalsize,series=\bfseries}
\setbeamertemplate{navigation symbols}{}
\usecolortheme{dolphin}
\setbeamercovered{transparent} %pode ser: invisible
%\setbeamertemplate{mini frames}[tick]

%\setbeamertemplate{mini frames}[tick]

\usepackage[framemethod=tikz]{mdframed}

\definecolor{mycolor}{rgb}{0.122, 0.435, 0.698}

\newmdenv[innerlinewidth=0.5pt, roundcorner=4pt,linecolor=mycolor,innerleftmargin=6pt,
innerrightmargin=6pt,innertopmargin=6pt,innerbottommargin=6pt]{mybox}

%Para inserir rodapé
\setbeamertemplate{footline}
{
  \leavevmode%
  \hbox{%
  \begin{beamercolorbox}[wd=.3\paperwidth,ht=2.25ex,dp=1ex,center]{subsection in head/foot}%
    \usebeamerfont{author in head/foot}\inserttitle
  \end{beamercolorbox}%
  \begin{beamercolorbox}[wd=.63\paperwidth,ht=2.25ex,dp=1ex,left,leftskip=0.5em]{title in head/foot}%
    \usebeamerfont{title in head/foot}\insertsubtitle
  \end{beamercolorbox}%
  \begin{beamercolorbox}[wd=.07\paperwidth,ht=2.25ex,dp=1ex,center]{title in head/foot}%
    \usebeamerfont{title in head/foot} \insertframenumber{}/ \inserttotalframenumber
  \end{beamercolorbox}}%
  \vskip0pt%
}


% Standard packages
% \usepackage[english]{babel}
% \usepackage[latin1]{inputenc}
\usepackage{times}
\usepackage[T1]{fontenc}
\usepackage{hyperref}
\usepackage{listings}
% \usepackage{listingsutf8}
\usepackage{graphicx}
\usepackage{fancybox}
\usepackage{multirow}
\usepackage{adjustbox}
%\usepackage{xcolor,colortbl} % colorir celulas de tabelas: http://tex.stackexchange.com/questions/94799/how-do-i-color-table-columns-and-rows
%\usepackage{animate} % PNG Animado: http://www.ipgp.fr/~lucas/Contrib/animbeamer.html  /  convert Bubble_sort_animation.gif -scene 1 +adjoin Bubble_sort_animation_%0d.png

% Setup TikZ
\usepackage{lmodern}
\usepackage{xparse}
\usepackage{tikz}
\usetikzlibrary{arrows}
\tikzstyle{block}=[draw opacity=0.7,line width=1.4cm]
\usetikzlibrary{shapes.callouts} % referencias: http://tex.stackexchange.com/questions/5423/how-to-open-a-temporary-comics-like-balloon-in-a-beamer-slide
                                 %              http://tex.stackexchange.com/questions/83783/explanatory-bubbles-in-beamer
                                 %              http://tex.stackexchange.com/questions/38805/simple-speech-bubbles-arrows-or-balloon-like-shapes-in-beamer
\tikzset{
    invisible/.style={opacity=0,text opacity=0},
    visible on/.style={alt=#1{}{invisible}},
    alt/.code args={<#1>#2#3}{%
      \alt<#1>{\pgfkeysalso{#2}}{\pgfkeysalso{#3}} % \pgfkeysalso doesn't change the path
    },
}

\NewDocumentCommand{\mycallout}{r<> O{align=center, fill=cyan!20, opacity=0.8,text opacity=1} m m m m}{%
  \tikz[remember picture, overlay]
    \node[text width=#4, #2,visible on=<#1>, rounded corners,
    draw,rectangle callout,anchor=pointer,callout relative pointer={(#5)}] at (#3) {#6};
}


\newcommand{\tikzmark}[1]{\tikz[overlay,remember picture,baseline=-0.5ex] \node (#1) {};}

% Setup listings
\lstset{ %
%   inputencoding=utf8/latin1,
  %backgroundcolor=\color{white},   % choose the background color; you must add \usepackage{color} or \usepackage{xcolor}
  basicstyle=\scriptsize\ttfamily, % the size of the fonts that are used for the code
  breakatwhitespace=false,         % sets if automatic breaks should only happen at whitespace
  breaklines=true,                 % sets automatic line breaking
  captionpos=b,                    % sets the caption-position to bottom
  commentstyle=\color{gray},       % comment style
  %deletekeywords={...},            % if you want to delete keywords from the given language
  escapeinside={\%*}{*)},          % if you want to add LaTeX within your code
  %extendedchars=true,              % lets you use non-ASCII characters; for 8-bits encodings only, does not work with UTF-8
  frame=leftline,                   % adds a frame around the code
  keepspaces=true,                 % keeps spaces in text, useful for keeping indentation of code (possibly needs columns=flexible)
  keywordstyle=\color{blue},       % keyword style
  language=C++,                    % the language of the code
  %morekeywords={*,...},            % if you want to add more keywords to the set
  numbers=left,                    % where to put the line-numbers; possible values are (none, left, right)
  numbersep=6pt,                   % how far the line-numbers are from the code
  numberstyle=\tiny\color{red},    % the style that is used for the line-numbers
  stepnumber=1,                    % the step between two line-numbers. If it's 1, each line will be numbered
  rulecolor=\color{black},         % if not set, the frame-color may be changed on line-breaks within not-black text (e.g. comments (green here))
  showspaces=false,                % show spaces everywhere adding particular underscores; it overrides 'showstringspaces'
  showstringspaces=false,          % underline spaces within strings only
  showtabs=false,                  % show tabs within strings adding particular underscores
  stringstyle=\color{red},         % string literal style
  tabsize=2,                       % sets default tabsize to 2 spaces
  %title=\lstname,                  % show the filename of files included with \lstinputlisting; also try caption instead of title
  extendedchars=true,
  inputencoding=utf8,
  literate=
    {á}{{\'a}}1
    {à}{{\`a}}1
    {ã}{{\~a}}1
    {é}{{\'e}}1
    {ê}{{\^e}}1
    {í}{{\'i}}1
    {ó}{{\'o}}1
    {õ}{{\~o}}1
    {ú}{{\'u}}1
    {ü}{{\"u}}1
    {ç}{{\c{c}}}1
%   literate=%
%          {á}{{\'a}}1
%          {í}{{\'i}}1
%          {é}{{\'e}}1
%          {ý}{{\'y}}1
%          {ú}{{\'u}}1
%          {ó}{{\'o}}1
%          {ě}{{\v{e}}}1
%          {š}{{\v{s}}}1
%          {č}{{\v{c}}}1
%          {ř}{{\v{r}}}1
%          {ž}{{\v{z}}}1
%          {ď}{{\v{d}}}1
%          {ť}{{\v{t}}}1
%          {ň}{{\v{n}}}1                
%          {ů}{{\r{u}}}1
%          {Á}{{\'A}}1
%          {Í}{{\'I}}1
%          {É}{{\'E}}1
%          {Ý}{{\'Y}}1
%          {Ú}{{\'U}}1
%          {Ó}{{\'O}}1
%          {Ě}{{\v{E}}}1
%          {Š}{{\v{S}}}1
%          {Č}{{\v{C}}}1
%          {Ř}{{\v{R}}}1
%          {Ž}{{\v{Z}}}1
%          {Ď}{{\v{D}}}1
%          {Ť}{{\v{T}}}1
%          {Ň}{{\v{N}}}1                
%          {Ů}{{\r{U}}}1  
}

% Para que o codigo seja exibido com overlay, destacando partes em sequencia
% http://tex.stackexchange.com/questions/8384/how-to-make-overlay-still-work-inside-lstlisting-environment
% http://tex.stackexchange.com/questions/100522/highlight-listings-in-beamer
% http://tex.stackexchange.com/questions/45011/beamer-how-to-do-src-highlighting-when-src-is-split-with-only-command
\lstdefinestyle{highlight}{
  basicstyle=\scriptsize\ttfamily\color{black}\colorbox{yellow},
  commentstyle=\color{gray},
  keywordstyle=\color{blue},
  stringstyle=\color{red},
  backgroundcolor=\color{yellow},
}
\lstdefinestyle{overlay}{
  basicstyle=\scriptsize\ttfamily\color{black!40}, % the size of the fonts that are used for the code
  breakatwhitespace=false,         % sets if automatic breaks should only happen at whitespace
  breaklines=true,                 % sets automatic line breaking
  captionpos=b,                    % sets the caption-position to bottom
  commentstyle=\color{gray!40},       % comment style
  escapeinside={(*}{*)},          % if you want to add LaTeX within your code
  frame=leftline,                   % adds a frame around the code
  keepspaces=true,                 % keeps spaces in text, useful for keeping indentation of code (possibly needs columns=flexible)
  keywordstyle=\color{blue!40},       % keyword style
  language=C++,                    % the language of the code
  numbers=left,                    % where to put the line-numbers; possible values are (none, left, right)
  numbersep=6pt,                   % how far the line-numbers are from the code
  numberstyle=\tiny\color{red},    % the style that is used for the line-numbers
  stepnumber=1,                    % the step between two line-numbers. If it's 1, each line will be numbered
  rulecolor=\color{black},         % if not set, the frame-color may be changed on line-breaks within not-black text (e.g. comments (green here))
  showspaces=false,                % show spaces everywhere adding particular underscores; it overrides 'showstringspaces'
  showstringspaces=false,          % underline spaces within strings only
  showtabs=false,                  % show tabs within strings adding particular underscores
  stringstyle=\color{red!40},         % string literal style
  tabsize=2,                       % sets default tabsize to 2 spaces
  moredelim=**[is][\only<+>{\color{black}\lstset{style=highlight}}]{@}{@},
  moredelim=**[is][\only<1->{\color{black}\lstset{style=highlight}}]{@1}{@},
  moredelim=**[is][\only<2->{\color{black}\lstset{style=highlight}}]{@2}{@},
  moredelim=**[is][\only<3->{\color{black}\lstset{style=highlight}}]{@3}{@},
  moredelim=**[is][\only<4->{\color{black}\lstset{style=highlight}}]{@4}{@},
  moredelim=**[is][\only<5->{\color{black}\lstset{style=highlight}}]{@5}{@},
  moredelim=**[is][\only<6->{\color{black}\lstset{style=highlight}}]{@6}{@},
  moredelim=**[is][\only<7->{\color{black}\lstset{style=highlight}}]{@7}{@},
  moredelim=**[is][\only<8->{\color{black}\lstset{style=highlight}}]{@8}{@},
  moredelim=**[is][\only<9->{\color{black}\lstset{style=highlight}}]{@9}{@},
}

% Listening com destaque para comentarios (NAO ESTA FUNCIONANDO, TENTATIVA NA AULA 04
\newenvironment{lstlistingCMT}
  {\begin{lstlisting}[commentstyle=\color{red}\bfseries]}
  {\end{lstlisting}}


% Modificando espaçamento entre itens
%\newenvironment{myitemize}
%  {\begin{itemize}\setlength{\itemsep}{0.5\baselineskip}}
%  {\end{itemize}}
%
%\newenvironment{myenumerate}
%  {\begin{enumerate}\setlength{\itemsep}{0.5\baselineskip}}
%  {\end{enumerate}}

\newlength{\wideitemsep}
\setlength{\wideitemsep}{0.5\itemsep}
\addtolength{\wideitemsep}{6pt}
\let\olditem\item
\renewcommand{\item}{\setlength{\itemsep}{\wideitemsep}\olditem}



% Setup links
\definecolor{links}{HTML}{2A1B81}
\hypersetup{colorlinks,linkcolor=,urlcolor=links}


%Equacao bold em section
\newcommand{\eq}[1]{\texorpdfstring{$#1$}{}}

%Equacao bold fora de section
\usepackage{bm}
\newcommand{\eqbf}[1]{$\bm{#1}$} %\(\mathbf{#1}\)

%Alert bold
\newcommand{\alertbf}[1]{\alert{\textbf{#1}}}

%Inserir link para animação
\newcommand{\animation}[1]{
  \vskip5mm
  \begin{center}
    \href{#1}{\includegraphics[scale=0.1]{setup/animacao.jpg} \\ \footnotesize{Animação}}
  \end{center}
}

%Inserir link para vídeo
\newcommand{\video}[1]{
  \vskip5mm
  \begin{center}
    \href{#1}{\includegraphics[scale=0.1]{setup/animacao.jpg} \\ \footnotesize{Vídeo}}
  \end{center}
}

%Inserir link para animação e vídeo
\newcommand{\animationvideo}[2]{
  %\vskip5mm
  \begin{center}
    \begin{table}[h]
      \begin{tabular}{c c}
        \href{#1}{\includegraphics[scale=0.1]{setup/animacao}} & \href{#2}{\includegraphics[scale=0.1]{setup/animacao}} \\
        \href{#1}{\footnotesize{Animação}}                  & \href{#2}{\footnotesize{Vídeo}}
      \end{tabular}
    \end{table}
  \end{center}
}


% Tabelas
\newcommand{\mr}[2]{\multirow{#1}{*}{#2}}
\newcommand{\mc}[2]{\multicolumn{#1}{|c|}{#2}}



\institute[UFOP]{
%  \inst{1}%
\title[Autor]{ASN}

  Universidade Federal de Ouro Preto, UFOP
  \vskip0mm
  Departamento de Computação, DECOM
%  \vskip3mm
%   Website: \textbf{www.decom.ufop.br/anascimento}
  \vskip0mm
    Professor : \textbf{Anderson Almeida Ferreira}
  }

% Author, Title, etc.
\author[psilva]{\textbf{Felipe Braz Marques, Matheus Peixoto Ribeiro Vieira, Pedro Henrique Rabelo Leão de Oliveira}}
\title[BCC241]{BCC241 - Projeto e Análise de Algoritmos}
\date{
% \the\year{}
\vskip4mm
\includegraphics[scale=0.5]{setup/logos.jpg}
}

% Para inserir um frame de conteúdo antes de cada seção
\AtBeginSection[]{
    % \begin{frame}\frametitle{Conteúdo}
    %     \begin{small}
    %         \tableofcontents[currentsection]%, hideothersubsections] % PROCURAR SOLUCAO PARA APRESENTAR SUBSECOES
    %     \end{small}
    % \end{frame}
    \begin{frame}
        \vfill
        \centering
            \begin{beamercolorbox}[sep=8pt,center,shadow=true,rounded=true]{title}
            \usebeamerfont{title}\insertsectionhead\par%
        \end{beamercolorbox}
        \vfill
    \end{frame}
}

\newcommand{\begindocument}[1]{
  {\setbeamertemplate{footline}{} % Para não colocar rodapé na primeira página
  \begin{frame}[plain]%para não colocar barra de navegação na primeira página
    \titlepage
  \end{frame}
  }
  #1
  \begin{frame}{Conteúdo}
    \begin{small}
        \tableofcontents
    \end{small}
  \end{frame}
}

\addbibresource{setup/refs.bib}


\subtitle[PAA]  {Clique, Conjunto Independente e SAT}
\begin{document}
\begindocument

\section{Clique}
    \begin{frame}{Problema do Clique}
        \begin{justify}
            Dado um grafo, o objetivo é encontrar um conjunto máximo de vértices tal que todas as possíveis arestas entre eles estejam presentes. Para isso, foi utilizado a estratégia \textit{branch and bound} para encontrar a solução do problema.
        \end{justify}
    \end{frame}
    
    \begin{frame}{Definição do problema}
        \begin{itemize}
            \item \textbf{Variáveis para a solução:} \(X_1, \dots, X_n\), onde \(X_i\) representa um vértice pertencente ao grafo.
            \item \textbf{Domínio para as variáveis da solução:} \{0, 1\}, onde 0 indica que aquele vértice não faz parte do clique máximo e 1 indica o contrário.
            \item \textbf{Restrições:} Todos os vértices representados pelas variáveis que possuem valor 1 devem estar conectados entre si através de uma aresta.
            \item \textbf{Objetivo:} Obter o maior conjunto possível de vértices tal que todas as possíveis arestas entre eles estejam presentes.
        \end{itemize}
    \end{frame}

    \begin{frame}{Leitura do Problema}
        \begin{itemize}
            \item O problema foi lido e armazenado em uma única variável, contendo o número de vértices e a matriz de adjacência do grafo.
            \item O vetor que armazena a solução inicial foi iniciado com todas as posições possuindo valores -1, para representar que a mesma começa vazia.
        \end{itemize}
    \end{frame}

    \begin{frame}[fragile]{Geração Inicial da Melhor Solução}
        \begin{lstlisting}
def geraSolucao(problema):
    solucao = [0] * problema[0][0]

    for i in range(problema[0][0]):
        for j in range(problema[0][0]):
            if(problema[i+1][j] == 1):
                solucao[i] = 1
                solucao[j] = 1
                return solucao

    solucao[0] = 1
    return solucao
        \end{lstlisting}
    \end{frame}
    
    \begin{frame}[fragile]{Função que executa o branch and bound}
        \begin{lstlisting}
def branchAndBoundClique(solucao, i, problema, melhor):
    if eCompleta(solucao):
        melhor[:] = solucao
        return melhor
    
    else:
        for j in range(2):
            solucao[i] = j
            
            if(eConsistente(solucao, problema, i) and ePromissora(solucao, problema, melhor, i)):
                melhor = branchAndBoundClique(solucao, i+1, problema, melhor)
            
            solucao[i] = -1
        
        return melhor
        \end{lstlisting}
    \end{frame}

    \begin{frame}[fragile]{Função que verifica a consistência da solução}
        \begin{lstlisting}
def eConsistente(solucao, problema, i):
    verticesNaSolucao = []
    
    for j in range(i+1):
        if solucao[j] == 1:
            verticesNaSolucao.append(j)

    for j in range(len(verticesNaSolucao)-1):
        for k in range(j+1, len(verticesNaSolucao)):
            if(problema[verticesNaSolucao[j]+1][verticesNaSolucao[k]] == 0):
                return False
            
    return True 
        \end{lstlisting}
    \end{frame}

    \begin{frame}[fragile]{Função que verifica se a solução é promissora}
        \begin{lstlisting}
def ePromissora(solucao, problema, melhor, i):
    numVerticesMelhor = sum(melhor)
    numVerticesMaximoSolucao = 0 # armazena o numero maximo de vertices possiveis nessa solucao
    
    for j in range(i+1):
        numVerticesMaximoSolucao += solucao[j]

    numVerticesMaximoSolucao += (problema[0][0] - (i+1))
    
    return numVerticesMaximoSolucao > numVerticesMelhor
        \end{lstlisting}
    \end{frame}

    \begin{frame}
        \begin{figure}[H]
            \centering
            \includegraphics[width=0.65\textwidth]{images/grafo_instancia1.png}
            \caption{Grafo da instância de entrada.}
            \label{fig:instancia1Clique}
        \end{figure}
    \end{frame} 

    \begin{frame}
        \begin{tcolorbox}[title=Saída da instância de entrada, width=\linewidth, 
          fontupper=\ttfamily, 
          halign=flush left]
            [0, 0, 0, 0, 0, 0, 1, 1, 1] \\
            Vértices presentes no clique: \\
            7 \\
            8 \\
            9 \\
            Tempo de execução: 0.001000 segundos \\
        \end{tcolorbox}
    \end{frame}

\section{Conjunto Independente}
    \begin{frame}{O problema do Conjunto Independente}
    \begin{justify}
        Para resolver o problema do Conjunto Independente (ou Conjunto Estável) por meio de uma redução polinomial ao problema do Clique, iremos usar a relação entre esses dois problemas. Sabemos que um conjunto independente de um grafo é equivalente a um clique no complemento desse grafo.
    \end{justify}
    \end{frame}

    \begin{frame}{Passos para a solução}
        \begin{itemize}
            \item \textbf{Complemento do grafo:} O complemento de um grafo é um grafo onde as arestas que estavam presentes no grafo original são removidas e as arestas que não estavam presentes são adicionadas.
            \item \textbf{Redução:} Dado um grafo GGG, o problema do conjunto independente em GGG pode ser resolvido encontrandoo o clique máximo no complemento do grafo G'G'G'.
            \item \textbf{Aproveitar a implementação do clique:} Após calcular o complemento do grafo, aplicamos o algoritmo de clique já implementado.\\
        \end{itemize}
    \end{frame}

    \begin{frame}[fragile]{Função geraComplemento}
        \begin{lstlisting}
def geraComplemento(problema):
    numVertices = problema[0][0]  # Numero de vertices (primeira linha da matriz)
    
    # Inicializa a matriz de complemento
    complemento = [[0] * numVertices for _ in range(numVertices)]
    
    # Preenche a matriz de complemento
    for i in range(numVertices):
        for j in range(numVertices):
            if i != j:  # Nao considerar a diagonal principal
                complemento[i][j] = 1 - problema[i + 1][j]
    
    # Adiciona o numero de vertices como a primeira linha da matriz de complemento
    complemento.insert(0, [numVertices])
    
    return complemento
        \end{lstlisting}
    \end{frame}
    \begin{frame}
        \begin{figure}[H]
            \centering
            \includegraphics[width=0.65\textwidth]{images/grafo_instancia1.png}
            \caption{Grafo da instância de entrada.}
            \label{fig:instancia1}
        \end{figure}
    \end{frame} 

    \begin{frame}
        \begin{figure}[H]
            \centering
            \includegraphics[width=0.65\textwidth]{images/conj_instancia1.png}
            \caption{Grafo complemento da instância de entrada.}
            \label{fig:instancia2}
        \end{figure}
    \end{frame}
    
    \begin{frame}{Saída}
        \begin{tcolorbox}[title=Saída da instância, width=\linewidth, 
          fontupper=\ttfamily, 
          halign=flush left]
            Conjunto independente máximo: {1, 2, 4, 6, 9}\\
            Tempo de execução: 0.002493 segundos\\
        \end{tcolorbox}
    \end{frame}

\section{SAT}
    \begin{frame}{O problema SAT}
        \begin{justify}
            Dada uma fórmula booleana na forma normal conjuntiva, é necessário encontrar, com \textit{backtracking}, algum resultado que satisfaça a mesma. 
            Caso não seja possível, informa tal possibilidade
        \end{justify}
    \end{frame}
    \begin{frame}{Definição do problema}
        \begin{itemize}
            \item \textbf{Variáveis para a solução:} \(X_1, \dots, X_n\), onde \(X_i\) indica o valor verdade para determinada variável do problema;
            \item \textbf{Domínio para as variáveis da solução:} \{False, True\};
            \item \textbf{Restrições:} Todas as cláusulas devem resultar em verdadeiro
            \item \textbf{Objetivo:} Obter algum conjunto de valores do domínio para as variáveis de forma que todas as cláusulas sejam satisfeitas
        \end{itemize}
    \end{frame}
    \begin{frame}{Entrada}
        \[(A \lor \neg B \lor C) \land (\neg A \lor B \lor D) \land (B \lor \neg C \lor \neg D) \land (\neg A \lor \neg B \lor D)\]
        \begin{tcolorbox}[title=Arquivo de entrada para a fórmula, width=\linewidth, fontupper=\ttfamily, halign=flush left]
            4 \\
            1 0 1 -1 \\
            0 1 -1 1 \\
            -1 1 0 0 \\
            0 0 -1 1
        \end{tcolorbox}
    \end{frame}

    \begin{frame}[fragile]{Backtracking}
        \begin{lstlisting}
def backtrack(solucao, i, problema):
    dominio = [False, True]
    if verificar_solucao(problema, solucao):
        return True
    else:
        i = i + 1
        candidatos = construir_candidatos(solucao, i, problema, dominio)
        for c in candidatos:
            solucao[i-1] = c
            finished = backtrack(solucao, i, problema)
            if finished:
                return True
            solucao[i-1] = None
    return False
        \end{lstlisting}
    \end{frame}

    \begin{frame}{Construir candidatos}
        \begin{itemize}
            \item Iterar sobre cada valor do domínio e colocá-lo na solução
            \item Verificar cada cláusula do problema
            \item Verificar se as variáveis já atribuídas compõem a cláusula
            \item Se comporem, verificar se a cláusula é satisfeita. 
            \item Se todas as cláusulas forem satisfeitas, o valor pode ser suficiente.
            \item Se uma cláusula não for, tal valor do domínio não poderá participar da solução.
        \end{itemize}
    \end{frame}

    \begin{frame}[fragile]{Construir candidatos}
        \begin{lstlisting}
def construir_candidatos(solucao, i, problema, dominio):
    if i > len(solucao): return []
    candidatos = dominio.copy()
    s_temp = solucao.copy()
    qtd_variaveis = len(problema[0])
    for c in dominio:
        s_temp[i-1] = c
        for clausula in problema:
            satisfeita = False  
            if clausula_computavel(i, clausula, qtd_variaveis):
                for variavel_atual, literal in enumerate(clausula):
                    if literal == 1:
                        satisfeita = satisfeita or s_temp[variavel_atual]
                    elif literal == 0:
                        satisfeita = satisfeita or not s_temp[variavel_atual]
            else:continue
            if not satisfeita:
                candidatos.remove(c)
                break
    return candidatos
        \end{lstlisting}
    \end{frame}

    \begin{frame}[fragile]{Cláusula computável}
        \begin{itemize}
            \item A cláusula a seguir não poderá ser computável.
            \item Estamos analisando o valor do True e a cláusula ficará falsa.
            \item Não sabemos se ela será verdadeira ou falsa no fim do processamento.
        \end{itemize}
        \begin{tcolorbox}[width=\linewidth, fontupper=\ttfamily,  halign=flush left]
            Cláusula: 1 -1 0 0 1 \newline 
            Solução: [False, False, True, \_, \_] \newline
            i = 3
        \end{tcolorbox}
        \begin{lstlisting}
def clausula_computavel(i, clausula, qtd_variaveis):  
    for j in range(i, qtd_variaveis):
        if clausula[j-1] == -1:
            return False
    return True
        \end{lstlisting}
    \end{frame}


    \begin{frame}{Exemplo instância}
        \[(A \lor B \lor C) \land (A \lor \neg B) \land (B \lor \neg C) \land (\neg A \lor C) \land (\neg A \lor \neg B \lor \neg C)\]
        
        \begin{tcolorbox}[title=Saída da instância, width=\linewidth, fontupper=\ttfamily, halign=flush left]
            Solução não encontrada \\
            Tempo de execução: 0.008312 segundos  
        \end{tcolorbox}
    \end{frame}
    %%%%%%%%%%%%%%%%%%%%%%%%%%%%%%%%%%%%%%%%%%%%%%
    \begin{frame}{Exemplo instância}
        \begin{center}
            \((A \lor B \lor C) \land (\neg A \lor \neg B \lor \neg C) \land (A \lor \neg B \lor D) \land (\neg A \lor B \lor \neg D) \land (A \lor \neg C \lor D) \land \)

            \((\neg A \lor C \lor \neg D) \land (\neg A \lor \neg B \lor C \lor D) \land (A \lor B \lor \neg C \lor \neg D)\)
            
        \end{center}
        
        \begin{tcolorbox}[title=Saída da instância, width=\linewidth, fontupper=\ttfamily, halign=flush left]
            Solução encontrada: [True, False, False, False] \\
            Tempo de execução: 0.000695 segundos
        \end{tcolorbox}
    \end{frame}
    %%%%%%%%%%%%%%%%%%%%%%%%%%%%%%%%%%%%%%%%%%%%%%
    \begin{frame}{Exemplo instância}
        \begin{tcolorbox}[title=Entrada da instância, width=\linewidth, fontupper=\ttfamily,  halign=flush left]
            Variáveis: 18 \\
            Cláusulas: 4096
        \end{tcolorbox}
        \begin{tcolorbox}[title=Saída da instância, width=\linewidth, fontupper=\ttfamily, halign=flush left]
            Solução encontrada: [False, False, False, False, False, False, False, False, False, False, False, False, False, False, False, True, False, False] \\
            Tempo de execução: 0.168715 segundos
        \end{tcolorbox}
    \end{frame}
    %%%%%%%%%%%%%%%%%%%%%%%%%%%%%%%%%%%%%%%%%%%%%%
    \begin{frame}{Exemplo instância}
        \begin{tcolorbox}[title=Entrada da instância, width=\linewidth, fontupper=\ttfamily,  halign=flush left]
            Variáveis: 10 \\
            Cláusulas: 59049
        \end{tcolorbox}
        \begin{tcolorbox}[title=Saída da instância, width=\linewidth, fontupper=\ttfamily, halign=flush left]
            Solução não encontrada \\
            Tempo de execução: 0.205896 segundos
        \end{tcolorbox}
    \end{frame}



\end{document}
